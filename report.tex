% Options for packages loaded elsewhere
\PassOptionsToPackage{unicode}{hyperref}
\PassOptionsToPackage{hyphens}{url}
%
\documentclass[
  12pt,
]{article}
\usepackage{amsmath,amssymb}
\usepackage{iftex}
\ifPDFTeX
  \usepackage[T1]{fontenc}
  \usepackage[utf8]{inputenc}
  \usepackage{textcomp} % provide euro and other symbols
\else % if luatex or xetex
  \usepackage{unicode-math} % this also loads fontspec
  \defaultfontfeatures{Scale=MatchLowercase}
  \defaultfontfeatures[\rmfamily]{Ligatures=TeX,Scale=1}
\fi
\usepackage{lmodern}
\ifPDFTeX\else
  % xetex/luatex font selection
\fi
% Use upquote if available, for straight quotes in verbatim environments
\IfFileExists{upquote.sty}{\usepackage{upquote}}{}
\IfFileExists{microtype.sty}{% use microtype if available
  \usepackage[]{microtype}
  \UseMicrotypeSet[protrusion]{basicmath} % disable protrusion for tt fonts
}{}
\makeatletter
\@ifundefined{KOMAClassName}{% if non-KOMA class
  \IfFileExists{parskip.sty}{%
    \usepackage{parskip}
  }{% else
    \setlength{\parindent}{0pt}
    \setlength{\parskip}{6pt plus 2pt minus 1pt}}
}{% if KOMA class
  \KOMAoptions{parskip=half}}
\makeatother
\usepackage{xcolor}
\usepackage[left=1cm, top=1cm, right=1cm, bottom=2cm]{geometry}
\usepackage{listings}
\newcommand{\passthrough}[1]{#1}
\lstset{defaultdialect=[5.3]Lua}
\lstset{defaultdialect=[x86masm]Assembler}
\usepackage{longtable,booktabs,array}
\usepackage{calc} % for calculating minipage widths
% Correct order of tables after \paragraph or \subparagraph
\usepackage{etoolbox}
\makeatletter
\patchcmd\longtable{\par}{\if@noskipsec\mbox{}\fi\par}{}{}
\makeatother
% Allow footnotes in longtable head/foot
\IfFileExists{footnotehyper.sty}{\usepackage{footnotehyper}}{\usepackage{footnote}}
\makesavenoteenv{longtable}
\usepackage{graphicx}
\makeatletter
\def\maxwidth{\ifdim\Gin@nat@width>\linewidth\linewidth\else\Gin@nat@width\fi}
\def\maxheight{\ifdim\Gin@nat@height>\textheight\textheight\else\Gin@nat@height\fi}
\makeatother
% Scale images if necessary, so that they will not overflow the page
% margins by default, and it is still possible to overwrite the defaults
% using explicit options in \includegraphics[width, height, ...]{}
\setkeys{Gin}{width=\maxwidth,height=\maxheight,keepaspectratio}
% Set default figure placement to htbp
\makeatletter
\def\fps@figure{htbp}
\makeatother
\setlength{\emergencystretch}{3em} % prevent overfull lines
\providecommand{\tightlist}{%
  \setlength{\itemsep}{0pt}\setlength{\parskip}{0pt}}
\setcounter{secnumdepth}{-\maxdimen} % remove section numbering
% Contents of listings-setup.tex
\usepackage{xcolor}

\lstset{
    basicstyle=\ttfamily,
    numbers=left,
    numberstyle=\footnotesize,
    stepnumber=1,
    numbersep=4pt,
    backgroundcolor=\color{black!10},
    showspaces=false,
    showstringspaces=false,
    showtabs=false,
    tabsize=2,
    captionpos=b,
    breaklines=true,
    breakatwhitespace=true,
    breakautoindent=true,
    linewidth=\textwidth
}
\ifLuaTeX
  \usepackage{selnolig}  % disable illegal ligatures
\fi
\IfFileExists{bookmark.sty}{\usepackage{bookmark}}{\usepackage{hyperref}}
\IfFileExists{xurl.sty}{\usepackage{xurl}}{} % add URL line breaks if available
\urlstyle{same}
\hypersetup{
  pdftitle={Sous-Chef AI},
  pdfauthor={Pietro Masolini 30613A},
  hidelinks,
  pdfcreator={LaTeX via pandoc}}

\title{Sous-Chef AI}
\author{Pietro Masolini 30613A}
\date{a.a. 2025/2026}

\begin{document}
\maketitle

{
\setcounter{tocdepth}{3}
\tableofcontents
}
\includegraphics{./assets/logo.png}

\hypertarget{sous-chef-ai}{%
\section{Sous-Chef AI}\label{sous-chef-ai}}

Sistema automatizzato per la ricerca, la costruzione, la revisione ed il
rilascio di manuali d'uso.

\begin{quote}
Sous-Chef AI, dove lo Chef sei tu!
\end{quote}

\hypertarget{introduzione}{%
\subsection{Introduzione}\label{introduzione}}

Il progetto ha due soggetti principali:

\begin{itemize}
\tightlist
\item
  Il \textbf{sistema}, ovvero il generatore di manuali
\item
  Il \textbf{manuale}, ovvero l'output del sistema
\end{itemize}

Il \textbf{sistema} mira a semplificare il processo di realizzazione di
manuali fornendo all'utenza:

\begin{itemize}
\tightlist
\item
  \emph{Ricerca automatizzata} all'interno del documentale
\item
  \emph{Traduzione e adattamento} del contenuto seguendo le indicazioni
  dell'utente
\item
  \emph{Processo di revisione c}he permette la modifica del contenuto o
  la sua completa riscrittura, specificando un prompt aggiuntivo
\item
  \emph{Pubblicazione automatica} su
  \href{https://roccobalocco.github.io/edu_digital/}{GitHubPages} di
  tutti i file legati al manuale
\item
  \emph{Creazione del manuale}, composto da più sezioni generate e
  revisionate singolarmente, in vari formati
\end{itemize}

Il \textbf{manuale}, tramite linguaggio non tecnico e coerente con il
contesto aziendale, mira a spiegare e comprendere:

\begin{itemize}
\tightlist
\item
  \emph{Come utilizzare l'IA} all'interno dell'azienda
\item
  \emph{Come adattare istruzioni operative} ai diversi casi d'uso
\item
  \emph{Esempi pratici} contestualizzati
\item
  \emph{Best practices e limiti} dell'IA
\end{itemize}

\hypertarget{ideazione}{%
\subsection{Ideazione}\label{ideazione}}

\hypertarget{tema}{%
\subsubsection{Tema}\label{tema}}

Il \textbf{sistema} è stato progettato per essere quanto più generale
possibile, per questo le tematiche che ricerca sono elencate alla voce
\passthrough{\lstinline!TOPIC\_SPECS!}. L'utenza può modellare le
tematiche, utilizzando il seguente schema:

\begin{lstlisting}
"topic_title": {
    "title": "Title to include",
    "query": "Query for LLM",
    "deliverable_sections": [
        "Section1",
        "Section2",
        "Example1",
        "Example2",
        "Pros/Cons",
    ]
}
\end{lstlisting}

Anche il processo di selezione dei documenti particolarmente rilevanti è
personalizzabile tramite la voce
\passthrough{\lstinline!TOPIC\_SELECTOR\_SYSTEM!}, la quale rappresenta
il prompt di sistema utilizzato selezionare i documenti migliori
mediante FAISS.

I temi principali su cui il l'\textbf{utente utilizzatore} si concentra
diventano quindi:

\begin{itemize}
\tightlist
\item
  \emph{Una efficace scrittura dei prompt} , sia in fase di estrazione
  che di selezione
\item
  \emph{Il controllo e la revisione dei contenuti generati}, che
  consente un ulteriore fine tuning mediante prompt aggiuntivi
\end{itemize}

Mentre il \textbf{sistema} concentra l'attenzione su tre aspetti
principali:

\begin{itemize}
\tightlist
\item
  \emph{Automazione della produzione dei contenuti}, che velocizza la
  creazione del manuale e riduce il carico cognitivo dell'utente nella
  fase di ricerca.
\item
  \emph{Revisione umana,} necessaria per assicurare accuratezza,
  coerenza e aderenza alle specifiche editoriali.
\item
  \emph{Adattamento dei contenuti}, che consente di personalizzare il
  manuale in base ai feedback e alle esigenze specifiche di ciascun
  argomento.
\end{itemize}

La generazione dei contenuti \emph{richiede un intervento attivo da
parte dell'utenza} per adattare i temi alle specifiche necessità del
manuale. Ogni sezione deve essere sottoposta ad una revisione rigida e
rigorosa, che garantisce la coerenza e la qualità dei contenuti.

L'\textbf{utente} può dunque concentrarsi minormente sulla ricerca del
contenuto, dedicando più tempo alla revisione e alle specifiche
richieste editoriali.

Il \textbf{manuale} invece si concentra su contenuti concreti e
fruibili, pensati per guidare i destinatari nell'uso dell'IA in contesti
reali.

I temi principali che questo ultimo affronta sono:

\begin{itemize}
\tightlist
\item
  \emph{Introduzione all'uso dell'IA in contesti lavorativi}, con un
  inquadramento pratico del ruolo dell'intelligenza artificiale nelle
  attività quotidiane di copywriter e content strategist, evidenziandone
  potenzialità, limiti operativi e necessità di supervisione umana
\item
  \emph{Istruzioni operative per casi d'uso specifici}, con sezioni
  dedicate alla costruzione e all'utilizzo di prompt efficaci per
  attività quali:

  \begin{itemize}
  \tightlist
  \item
    Text generation, per la creazione di contenuti testuali destinati a
    post, newsletter o articoli
  \item
    Language detection, per l'identificazione automatica della lingua
    dei contenuti
  \item
    Cross-tabular analysis, per l'analisi di tabelle e dati testuali
    finalizzata ad individuare pattern, anomalie e insight
  \end{itemize}
\item
  \emph{Esempi pratici contestualizzati}, selezionati dal repository
  OpenAI Cookbook e adattati a scenari realistici di agenzia e azienda
\item
  \emph{Best practices e limiti d'uso}, con indicazioni per evitare
  errori comuni nell'utilizzo degli LLM, migliorare i prompt, verificare
  gli output e integrare i risultati
\end{itemize}

In sintesi, mentre il \textbf{sistema} si concentra sull'automazione, la
selezione e la generazione dei contenuti, il \textbf{manuale} offre una
guida pratica e contestualizzata che consente all'utente di applicare
efficacemente l'IA, mantenendo il controllo creativo e strategico sui
contenuti prodotti.

\hypertarget{destinatari}{%
\subsubsection{Destinatari}\label{destinatari}}

\textbf{Sous-Chef AI} ha lo scopo di supportare il creatore del manuale,
semplificando e automatizzando gran parte del flusso editoriale.
L'utente può quindi indicizzare, selezionare, adattare e revisionare
contenuti a partire dall'OpenAI Cookbook concentrandosi sul controllo
qualitativo.

In questo caso il \textbf{manuale} é pensato per copywriter e content
strategist all'interno di una agenzia di comunicazione digitale. Esso
fornisce linee guida pratiche, esempi contestualizzati e buone pratiche
nell'uso dell'IA generativa, utilizzando un linguaggio accessibile e
colloquiale.

\hypertarget{personas}{%
\paragraph{Personas}\label{personas}}

\includegraphics[width=0.9\textwidth,height=\textheight]{./assets/alice.png}

\textbf{Scenario d'uso}: deve preparare un calendario editoriale e
utilizza il manuale per capire come generare contenuti coerenti e adatti
al tono aziendale.

\includegraphics{./assets/marco.png}

\textbf{Scenario d'uso}: deve identificare delle best practices per
generare testi coerenti e personalizzati, consultando il manuale per
selezionare/creare prompt efficaci e comprendere i limiti degli
strumenti AI.

\includegraphics{./assets/giulia.png}

\textbf{Scenario d'uso}: usa il manuale per definire standard condivisi,
monitorare le pratiche del team e introdurre nuovi strumenti in modo
controllato.

\hypertarget{requisiti-di-accettazione}{%
\subsubsection{Requisiti di
accettazione}\label{requisiti-di-accettazione}}

Per raggiungere efficacemente i rispettivi destinatari, il
\textbf{manuale} e il \textbf{sistema}, devono soddisfare una serie di
richieste, le quali spaziano su diverse aree.

\hypertarget{requisiti-funzionali}{%
\paragraph{Requisiti funzionali}\label{requisiti-funzionali}}

\begin{itemize}
\tightlist
\item
  \emph{Completezza dei contenuti}, il manuale deve includere istruzioni
  operative, esempi e linee guida sui limiti d'uso, coprendo almeno i
  casi d'uso selezionati
\item
  \emph{Chiarezza e comprensibilità}, il linguaggio deve essere fruibile
  da personale non tecnico e deve rimanere coerente con il contesto
  aziendale
\item
  \emph{Modularità}, ogni sezione deve risultare indipendente e
  integrabile senza compromettere la coerenza complessiva
\item
  \emph{Validazione umana}, tutte le sezioni devono passare attraverso
  un processo di revisione editoriale, atto a garantire accuratezza,
  coerenza e aderenza alle specifiche
\item
  \emph{Per utilizzare mdBook} è richiesto

  \begin{itemize}
  \tightlist
  \item
    un file \passthrough{\lstinline!book.toml!} dove definire metadati e
    configurazioni essenziali (eg: directory di output)
  \item
    un file \passthrough{\lstinline!SUMMARY.md!} dove definire la
    struttura e la navigazione del contenuto
  \end{itemize}
\end{itemize}

\hypertarget{modelli-di-fruizione}{%
\paragraph{Modelli di fruizione}\label{modelli-di-fruizione}}

\begin{itemize}
\tightlist
\item
  \emph{Non lineare e modulare}, i destinatari possono accedere
  direttamente alle sezioni di cui hanno bisogno, senza dover leggere
  l'intero manuale
\item
  \emph{Accesso multicanale}, il manuale è disponibile in diversi
  formati digitali, i quali supportano dispositivi e contesti differenti
\end{itemize}

\hypertarget{aspetti-innovativi}{%
\paragraph{Aspetti innovativi}\label{aspetti-innovativi}}

\begin{itemize}
\tightlist
\item
  \emph{Human in the loop}, combinazione di generazione automatica con
  revisione umana per massimizzare efficienza e qualità
\item
  \emph{Personalizzazione}, possibilità, per l'utenza del sistema, di
  adattare prompt e sezioni ai casi d'uso specifici, creando manuali su
  misura
\item
  \emph{Automazione del flusso editoriale}, gestione automatica di
  indicizzazione, selezione e pubblicazione dei contenuti, riducendo in
  questo modo il carico operativo
\item
  \emph{Flessibilità nei canali di distribuzione}, generazione di output
  multi formato compatibili con web, intranet, stampa e multi
  dispositivo, semplificandone la diffusione
\end{itemize}

\hypertarget{canali-di-distribuzione}{%
\subsubsection{Canali di distribuzione}\label{canali-di-distribuzione}}

\hypertarget{canali-principali}{%
\paragraph{Canali principali:}\label{canali-principali}}

\begin{longtable}[]{@{}
  >{\raggedright\arraybackslash}p{(\columnwidth - 6\tabcolsep) * \real{0.1512}}
  >{\raggedright\arraybackslash}p{(\columnwidth - 6\tabcolsep) * \real{0.3488}}
  >{\raggedright\arraybackslash}p{(\columnwidth - 6\tabcolsep) * \real{0.1512}}
  >{\raggedright\arraybackslash}p{(\columnwidth - 6\tabcolsep) * \real{0.3488}}@{}}
\toprule\noalign{}
\begin{minipage}[b]{\linewidth}\raggedright
Canale
\end{minipage} & \begin{minipage}[b]{\linewidth}\raggedright
Descrizione
\end{minipage} & \begin{minipage}[b]{\linewidth}\raggedright
Formati
\end{minipage} & \begin{minipage}[b]{\linewidth}\raggedright
Note
\end{minipage} \\
\midrule\noalign{}
\endhead
\bottomrule\noalign{}
\endlastfoot
\textbf{Web/Intranet} & Manuale accessibile via intranet, portale
aziendale o link pubblico & HTML, PDF, md (per mdBook) & Navigazione
modulare con link interni e indici interattivi, ricerca integrata
(mdBook) \\
\textbf{Social} & Sintesi o estratti mirati per formazione rapida & PDF,
snippet & Contenuti ridotti per fruizione veloce, non completo \\
\textbf{Repository/Marketplace} & Condivisione con altri team o agenzie
& Markdown, TeX & Versione ``template'' riutilizzabile per progetti
futuri \\
\end{longtable}

\hypertarget{formati}{%
\paragraph{Formati:}\label{formati}}

Il \textbf{manuale} risulta disponibile nei seguenti formati per
garantire massima flessibilità:

\begin{itemize}
\tightlist
\item
  \textbf{\href{https://github.com/roccobalocco/edu_digital/releases/download/docs-20260110175016/manuale_completo.md}{Markdown}}:
  formato sorgente, sia via moduli sia monolitico, facilmente
  aggiornabile e riutilizzabile.
\item
  \textbf{\href{https://github.com/roccobalocco/edu_digital/releases/download/docs-20260110175016/manuale_completo.html}{HTML}
  / \href{https://roccobalocco.github.io/edu_digital/}{mdBook}}:
  fruizione online con navigazione interattiva.
\item
  \textbf{\href{https://github.com/roccobalocco/edu_digital/releases/download/docs-20260110175016/manuale_completo.pdf}{PDF}}:
  stampa e distribuzione offline, con layout coerente e professionale.
\item
  \textbf{\href{https://github.com/roccobalocco/edu_digital/releases/download/docs-20260110175016/manuale_completo.epub}{ePub}}:
  lettura su dispositivi mobili e reader digitali.
\item
  \textbf{\href{https://github.com/roccobalocco/edu_digital/releases/download/docs-20260110175016/manuale_completo.tex}{Tex
  / LaTeX}}: generazione di versioni professionali tipograficamente
  accurate.
\end{itemize}

\hypertarget{identituxe0-visuale}{%
\paragraph{Identità visuale}\label{identituxe0-visuale}}

Il \textbf{manuale} si concentra sulla leggibilità e sulla facilità di
gestione nel tempo. Il formato Markdown garantisce una base solida e
coerente per la formattazione, che può poi essere adattata liberamente
dall'utente.

La responsabilità di definire l'identità visiva è lasciata agli
utilizzatori del \textbf{sistema}. In futuro potrebbe essere possibile
esportare i contenuti applicando stili, template, layout e altre
personalizzazioni anche grazie a Pandoc.

\hypertarget{tradizione-vs-innovazione}{%
\paragraph{Tradizione vs Innovazione}\label{tradizione-vs-innovazione}}

Il \textbf{manuale} mantiene coerenza con modelli consolidati di
documentazione digitale (ad esempio guide operative, manuali tecnici in
md/pdf/html). Allo stesso tempo il \textbf{sistema} introduce elementi
innovativi, come workflow automatizzati, human-in-the-loop e
personalizzazione dei contenuti, per offrire una fruizione dinamica.

\hypertarget{processo-di-produzione}{%
\subsection{Processo di Produzione}\label{processo-di-produzione}}

Il processo di produzione del \textbf{manuale} combina l'acquisizione
dei contenuti con la loro gestione, garantendo coerenza, tracciabilità e
aggiornabilità.

Il \textbf{sistema} è progettato per utilizzare fonti in formato Jupyter
Notebook (\passthrough{\lstinline!.ipynb!}). Per cambiare documentale è
necessario modificare il percorso della variabile di ambiente
\passthrough{\lstinline!COOKBOOK\_PATH!}. In base ai requisiti
progettuali, per questo manuale il percorso è impostato sul repository
OpenAI Cookbook.

Le fonti attraversano tre fasi principali di trasformazione:

\begin{enumerate}
\def\labelenumi{\arabic{enumi}.}
\tightlist
\item
  Inizialmente sono \emph{file} in formato
  \passthrough{\lstinline!.ipynb!}
\item
  Successivamente ne vengono selezionati e adattati alcuni tramite l'uso
  di un LLM
\item
  Infine, grazie ad una fase di revisione manuale, arrivano ad essere
  \emph{sezioni del manuale}
\end{enumerate}

Il flusso di gestione documentale segue queste fasi principali:

\begin{enumerate}
\def\labelenumi{\arabic{enumi}.}
\tightlist
\item
  \emph{Indicizzazione e raccolta dei contenuti}, le fonti vengono
  organizzate in un vector store per un recupero rapido e mirato
\item
  \emph{Selezione dei documenti rilevanti}, tramite prompt guidati e
  criteri definiti dall'utente
\item
  \emph{Adattamento e generazione delle sezioni}, trasformazione dei
  contenuti in sezioni Markdown
\item
  \emph{Revisione editoriale}, modifica, approvazione o rigenerazione
  delle sezioni basata su feedback dell'utente
\item
  \emph{Commit e versionamento}, le sezioni approvate vengono salvate
  tramite git una volta fatto il commit
\item
  \emph{Pubblicazione e esportazione}, tramite GitHub Actions, il
  contenuto viene distribuito su GitHub Pages e reso disponibile in
  diversi formati (HTML, PDF, ePub, TeX, Markdown)
\end{enumerate}

Il seguente diagramma mostra graficamente come si articola il flusso di
produzione, di gestione e di pubblicazione dei contenuti:

\hypertarget{tecnologie-adottate}{%
\subsubsection{Tecnologie adottate}\label{tecnologie-adottate}}

Il progetto integra diverse tecnologie, ciascuna con un ruolo specifico
nelle fasi di produzione del manuale, al fine di garantire efficienza,
coerenza e qualità dei contenuti.

\begin{itemize}
\item
  \emph{Python 3.12.1}, linguaggio di programmazione principale per lo
  sviluppo del sistema
\item
  \emph{LangGraph}, utilizzato per modellare il flusso di generazione
  dei contenuti (indicizzazione, recupero, selezione, adattamento e
  revisione). Contribuisce a rendere il processo \emph{modulare,
  tracciabile e ripetibile}, riducendo errori e ridondanze nella
  produzione
\item
  \emph{LangChain}, framework di orchestrazione dei LLM. Coordina
  l'interazione tra il sistema di gestione dei documenti e il modello
  generativo, supportando l'automazione delle fasi di selezione e
  adattamento dei contenuti
\item
  \emph{Google} \emph{Gemini}, modello generativo impiegato per la
  trasformazione dei contenuti selezionati in sezioni leggibili e
  coerenti con il contesto aziendale

  \begin{itemize}
  \tightlist
  \item
    \passthrough{\lstinline!env.GEMINI\_MODEL!}, utilizzato per la
    generazione dei testi (gemini-2.5-flash)
  \item
    \passthrough{\lstinline!env.GEMINI\_EMBED\_MODEL!}, utilizzato per
    creare embedding semantici delle fonti (text-embedding-004)
  \end{itemize}

  Contribuisce a \emph{velocizzare la produzione}, mantenendo un
  linguaggio non tecnico e adatto all'utenza
\item
  \emph{Vector store con FAISS ed embedding semantici}, permette
  l'indicizzazione delle fonti e il recupero rapido dei documenti più
  rilevanti. Prova a garantire che i contenuti generati siano
  \emph{accurati e contestualizzati}
\item
  \emph{GitHub Actions}, automatizzano le fasi di commit, pubblicazione
  e generazione del manuale in diversi formati (HTML, PDF, ePub, TeX,
  Markdown) mediante i seguenti workflow:

  \begin{itemize}
  \tightlist
  \item
    \href{https://github.com/roccobalocco/edu_digital/blob/docs-20260110175016/.github/workflows/conversion.yml}{conversion.yml},
    workflow che converte il manuale nei formati di destinazione,
    utilizzando Pandoc, per poi crearne un rilascio su GitHub
  \item
    \href{https://github.com/roccobalocco/edu_digital/blob/docs-20260110175016/.github/workflows/mdbook.yml}{mdbook.yml},
    workflow che aggiorna la GitHub Page legata al repository
  \end{itemize}

  Contribuiscono a rendere il flusso \emph{efficiente, ripetibile e
  facilmente aggiornabile}, supportando la distribuzione multi-canale
  dei contenuti
\end{itemize}

\hypertarget{esecuzione-del-flusso}{%
\subsubsection{Esecuzione del flusso}\label{esecuzione-del-flusso}}

Per eseguire il progetto, disponibile nel seguente
\href{https://github.com/roccobalocco/edu_digital}{repository}, è
necessario:

\begin{itemize}
\tightlist
\item
  \emph{copiare ed adattare il file}
  \passthrough{\lstinline!.env.example!} in
  \passthrough{\lstinline!.env!}, mettendo la propria chiave API ed il
  proprio percorso al suo interno
\item
  \emph{installare tutte le dipendenze} elencate in
  \passthrough{\lstinline!requirements.txt!} tramite il comando
  \passthrough{\lstinline!pip install -r requirements.txt!}
\item
  \emph{avere un documentale di file} \passthrough{\lstinline!.ipynb!}
  all'interno del percorso specificato nelle variabili di ambiente
\item
  \emph{eseguire il comando} \passthrough{\lstinline!python run.py!} con
  l'opzione \passthrough{\lstinline!--full!} per avere un file
  comprensivo di tutte le sezioni
\end{itemize}

\hypertarget{utilizzo-di-ia-generativa}{%
\subsubsection{Utilizzo di IA
generativa}\label{utilizzo-di-ia-generativa}}

L'IA generativa è stata integrata in varie fasi del flusso di gestione
documentale, con l'obiettivo di automatizzare, accelerare la produzione
del manuale.

\hypertarget{fasi-in-cui-lia-uxe8-stata-applicata}{%
\paragraph{Fasi in cui l'IA è stata
applicata}\label{fasi-in-cui-lia-uxe8-stata-applicata}}

\begin{enumerate}
\def\labelenumi{\arabic{enumi}.}
\tightlist
\item
  \emph{Selezione dei documenti rilevanti}

  \begin{itemize}
  \tightlist
  \item
    Viene utilizzato un modello di linguaggio per analizzare gli
    estratti provenienti dal vector store (FAISS) e identificare i
    contenuti più pertinenti per ciascun topic
  \item
    Obiettivo: ridurre il tempo che l'utente impiegherebbe a leggere e
    filtrare manualmente centinaia di notebook o estratti
  \end{itemize}
\item
  \emph{Adattamento e generazione delle sezioni}

  \begin{itemize}
  \tightlist
  \item
    L'IA genera sezioni del manuale in linguaggio coerente e non
    tecnico, adattando i contenuti selezionati al contesto aziendale
  \item
    Obiettivo: produrre rapidamente contenuti leggibili, coerenti e
    pronti per la revisione.
  \end{itemize}
\item
  \emph{Rigenerazione basata su feedback}

  \begin{itemize}
  \tightlist
  \item
    Durante la fase di revisione, l'utente può fornire indicazioni
    aggiuntive (prompt) che vengono utilizzate dal modello per
    migliorare o riformulare il contenuto
  \item
    Obiettivo: integrare il cosiddetto human-in-the-loop, combinando
    automazione e intervento umano per assicurare accuratezza e aderenza
    alle specifiche
  \end{itemize}
\end{enumerate}

\hypertarget{approccio-di-prompt-engineering}{%
\paragraph{Approccio di prompt
engineering}\label{approccio-di-prompt-engineering}}

I prompt sono suddivisi in due macro-categorie:

\begin{enumerate}
\def\labelenumi{\arabic{enumi}.}
\item
  \emph{Prompt di selezione}
  (\passthrough{\lstinline!TOPIC\_SELECTOR\_SYSTEM!}), utilizzati per
  guidare l'IA nell'identificazione degli estratti più pertinenti.
  Snippet di prompt:

\begin{lstlisting}[language=Python]
TOPIC_SELECTOR_SYSTEM = """
Sei un assistente che deve selezionare, da una lista di estratti, quelli più rilevanti
per un manuale aziendale su prompt engineering.
Scegli solo estratti utili e pratici; scarta parti irrilevanti.
Restituisci JSON con:
- selected: [{id, reason}]
- rejected: [{id, reason}]
"""
\end{lstlisting}
\item
  \emph{Prompt di generazione/adattamento}
  (\passthrough{\lstinline!ADAPT\_SYSTEM!}), impiegati per rielaborare i
  contenuti selezionati, definendone la struttura, lo stile ed il
  livello di dettaglio, in coerenza con gli obiettivi ed il contesto
  dato dall'utente. Snippet di prompt:

\begin{lstlisting}[language=Python]
MANUAL_STYLE_GUIDE = """
Sei un content strategist senior in un'agenzia di comunicazione digitale.
Obiettivo: creare un manuale d'uso dell'IA per copywriter e content strategist.
Stile:
- linguaggio non tecnico, concreto, orientato al lavoro quotidiano
- frasi brevi, punti elenco quando utile
- evita gergo da sviluppo (API, token, embeddings) se non indispensabile
- evidenzia: obiettivo, quando usarlo, come usarlo, esempi, limiti, buone pratiche
- se trovi concetti tecnici nel testo sorgente, riscrivi in modo semplice
Output: Markdown pulito, modulare, con sezioni coerenti.
"""

ADAPT_SYSTEM = MANUAL_STYLE_GUIDE + """
Regola importante: mantieni riferimenti alla fonte (path file + titolo se presente).
Non inventare contenuti; se manca un pezzo, segnalalo come 'Da integrare' senza inventare.
"""
\end{lstlisting}
\end{enumerate}

L'intero processo è concepito secondo un approccio human-in-the-loop,
non solo la fase di revisione, i prompt sono configurabili e
modificabili dall'utente del \textbf{sistema}, consentendo un
fine-tuning continuo basato su feedback editoriali e garantendo il
controllo umano sulle decisioni critiche che riguardano il flusso di
produzione.

\hypertarget{validazione-e-controllo-qualituxe0}{%
\paragraph{Validazione e controllo
qualità}\label{validazione-e-controllo-qualituxe0}}

Tutti gli output generati dall'IA passano attraverso una fase di
revisione umana. L'utente può approvare, modificare o rigenerare ogni
sezione, garantendo:

\begin{itemize}
\tightlist
\item
  \emph{Coerenza} tra i materiali
\item
  \emph{Accuratezza} delle informazioni
\item
  \emph{Aderenza} al contesto aziendale e alle specifiche richieste
\end{itemize}

\hypertarget{contributo-dellia}{%
\paragraph{Contributo dell'IA}\label{contributo-dellia}}

\hypertarget{riduzione-dei-tempi}{%
\subparagraph{Riduzione dei tempi}\label{riduzione-dei-tempi}}

La selezione automatica dei contenuti e la generazione delle sezioni
riducono drasticamente il lavoro manuale.

\hypertarget{miglioramento-della-qualituxe0}{%
\subparagraph{Miglioramento della
qualità}\label{miglioramento-della-qualituxe0}}

L'IA fornisce una base coerente e leggibile, su cui la revisione umana
può concentrarsi sul fine-tuning e sugli aspetti strategici.

\hypertarget{manutenibilituxe0}{%
\subparagraph{Manutenibilità}\label{manutenibilituxe0}}

L'utilizzo dell'IA facilita l'aggiornamento del manuale in caso di
variazioni delle fonti. Il \textbf{sistema} consente di rielaborare
rapidamente le sezioni interessate e di analizzare le modifiche
introdotte nei contenuti sorgente.

\hypertarget{limiti-e-intervento-umano}{%
\paragraph{Limiti e intervento umano}\label{limiti-e-intervento-umano}}

L'IA non è in grado di garantire completezza o correttezza totale delle
informazioni. La revisione umana rimane essenziale per:

\begin{itemize}
\tightlist
\item
  \emph{Contestualizzare} i contenuti al target aziendale
\item
  \emph{Adattare} esempi pratici
\item
  \emph{Validare} coerenza e accuratezza
\end{itemize}

\hypertarget{valutazione-dei-risultati-raggiunti}{%
\subsection{Valutazione dei risultati
raggiunti}\label{valutazione-dei-risultati-raggiunti}}

\hypertarget{valutazione-del-flusso-di-produzione}{%
\subsubsection{Valutazione del flusso di
produzione}\label{valutazione-del-flusso-di-produzione}}

Il flusso di produzione implementato in \textbf{Sous-Chef AI} ha
permesso di ottenere benefici in diverse fasi:

\begin{enumerate}
\def\labelenumi{\arabic{enumi}.}
\tightlist
\item
  \emph{Riduzione dei tempi di gestione documentale} grazie alla
  selezione automatica dei documenti tramite FAISS e la generazione di
  sezioni con Google Gemini
\item
  \emph{Riduzione degli errori}, la struttura modulare, il versionamento
  e la revisione guidata dall'utente limitano gli errori di coerenza e
  di contenuto
\item
  \emph{Miglioramento della qualità dei documenti}, l'uso combinato di
  LLM e revisione umana dovrebbe garantire testi coerenti con il
  contesto aziendale
\item
  \emph{Miglioramento del livello di accettazione della tecnologia}
  grazie alla possibilità di personalizzare i prompt e gestire i
  contenuti secondo le proprie esigenze
\item
  \emph{Raggiungimento di nuovi canali di distribuzione}, l'esportazione
  automatica dei contenuti, unita alla pubblicazione su GitHub Pages, ha
  permesso di raggiungere facilmente più canali di distribuzione
\item
  \emph{Soddisfacimento di nuovi scenari d'uso}, il sistema supporta
  scenari non previsti inizialmente, come l'adattamento rapido a nuovi
  contenuti
\end{enumerate}

\hypertarget{confronto-con-lo-stato-dellarte}{%
\subsubsection{Confronto con lo stato
dell'arte}\label{confronto-con-lo-stato-dellarte}}

\hypertarget{asis-flusso-tradizionale}{%
\paragraph{ASIS (flusso tradizionale)}\label{asis-flusso-tradizionale}}

Raccolta manuale dei contenuti, selezione e revisione esclusivamente
umane, trasformazione dei formati effettuata singolarmente.

\hypertarget{problemi-principali}{%
\subparagraph{Problemi principali}\label{problemi-principali}}

\begin{itemize}
\tightlist
\item
  \emph{Tempi lunghi}
\item
  \emph{Alto rischio di incoerenza tra sezioni}
\item
  \emph{Limitata scalabilità}
\item
  \emph{Difficoltà nel mantenere aggiornamenti frequenti}
\end{itemize}

\hypertarget{tobe-sous-chef-ai}{%
\paragraph{TOBE (Sous-Chef AI)}\label{tobe-sous-chef-ai}}

Pipeline automatizzata per selezione, adattamento e pubblicazione, con
revisione umana integrata.

\hypertarget{vantaggi-rilevanti}{%
\subparagraph{Vantaggi rilevanti}\label{vantaggi-rilevanti}}

\begin{itemize}
\tightlist
\item
  \emph{Riduzione dei tempi nelle fasi di ricerca e generazione}
\item
  \emph{Facilità di aggiornamento}
\item
  \emph{Modularità del contenuto}
\item
  \emph{Possibilità di pubblicare in più formati e canali in automatico}
\end{itemize}

\hypertarget{limiti-emersi}{%
\subsubsection{Limiti emersi}\label{limiti-emersi}}

\begin{itemize}
\tightlist
\item
  \emph{Accesso limitato ad alcune tecnologie}, alcuni LLM o embedding
  più avanzati potrebbero non essere disponibili o richiedere costi
  aggiuntivi
\item
  \emph{Automazione parziale dei formati}, la trasformazione completa in
  alcuni formati (es. TeX complesso o PDF con layout avanzato) può
  richiedere intervento manuale
\item
  \emph{Poco controllo sul layout}, non esiste, al momento, un modo per
  gestire questo aspetto del prodotto finale, se non intervenendo
  manualmente sul processo di conversione/generazione. Di seguito alcuni
  esempi di questo nei vari formati:

  \begin{itemize}
  \item
    Visualizzazione tramite mdBook:

    \includegraphics[width=0.8\textwidth,height=\textheight]{./assets/mdBook.png}
  \item
    Visualizzazione pdf:

    \includegraphics{./assets/pdf.png}
  \item
    Visualizzazione html:

    \includegraphics{./assets/html.png}
  \item
    Visualizzazione tex:

    \includegraphics{./assets/tex.png}
  \end{itemize}
\item
  \emph{Integrazione di fonti eterogenee}, il sistema è utilizzabile
  solo per fonti notebook \passthrough{\lstinline!.ipynb!}; formati
  diversi necessitano di adattamento preliminare
\end{itemize}

Gli ultimi due punti sono facilmente risolvibili all'interno di
eventuali future versioni di \textbf{Sous-Chef AI}.

\hypertarget{conclusioni}{%
\subsection{Conclusioni}\label{conclusioni}}

L'implementazione del \textbf{sistema} ha permesso di raggiungere i
seguenti obiettivi:

\begin{itemize}
\tightlist
\item
  \emph{Produzione rapida e modulare} di contenuti
\item
  \emph{Riduzione dei tempi} nella gestione documentale
\item
  \emph{Possibilità di raggiungere più canali} di distribuzione
\end{itemize}

I risultati più soddisfacenti riguardano l'efficienza della produzione,
grazie all'IA generativa e alla gestione automatizzata dei contenuti.

Le principali limitazioni restano legate alla alla gestione di fonti non
standard e alle trasformazioni avanzate di formato.

Tuttavia, il prototipo realizzato (interpretabile come un Proof of
Concept), dimostra chiaramente la fattibilità ed offre un'ottima base
per futuri ampliamenti della pipeline editoriale.

\hypertarget{bibliografia-sitografia-e-strumenti-utilizzati}{%
\subsection{Bibliografia, sitografia e strumenti
utilizzati}\label{bibliografia-sitografia-e-strumenti-utilizzati}}

\begin{itemize}
\tightlist
\item
  \href{https://chatgpt.com/}{ChatGPT} e
  \href{https://claude.ai/}{Claude} come supporto nella stesura del
  codice e nella validazione di questo report
\item
  Know-how ottenuto implementando
  \href{https://www.softeam.it/prodotti/sofia/}{SofIA} (LangGraph,
  embedding, Python, CD/CI, etc\ldots)
\item
  \href{https://github.com/features/codespaces}{GitHub Codespaces} per
  lo sviluppo in cloud
\item
  \href{https://app.creately.com/}{Creately} per la creazione delle
  personas cards
\item
  \href{github.com/roccobalocco/edu_digital/}{GitHub} per il repository
\item
  \href{https://sora.chatgpt.com}{Sora} per il logo del progetto
\end{itemize}

\end{document}
