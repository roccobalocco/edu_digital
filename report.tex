% Options for packages loaded elsewhere
\PassOptionsToPackage{unicode}{hyperref}
\PassOptionsToPackage{hyphens}{url}
\documentclass[
]{article}
\usepackage{xcolor}
\usepackage{amsmath,amssymb}
\setcounter{secnumdepth}{-\maxdimen} % remove section numbering
\usepackage{iftex}
\ifPDFTeX
  \usepackage[T1]{fontenc}
  \usepackage[utf8]{inputenc}
  \usepackage{textcomp} % provide euro and other symbols
\else % if luatex or xetex
  \usepackage{unicode-math} % this also loads fontspec
  \defaultfontfeatures{Scale=MatchLowercase}
  \defaultfontfeatures[\rmfamily]{Ligatures=TeX,Scale=1}
\fi
\usepackage{lmodern}
\ifPDFTeX\else
  % xetex/luatex font selection
\fi
% Use upquote if available, for straight quotes in verbatim environments
\IfFileExists{upquote.sty}{\usepackage{upquote}}{}
\IfFileExists{microtype.sty}{% use microtype if available
  \usepackage[]{microtype}
  \UseMicrotypeSet[protrusion]{basicmath} % disable protrusion for tt fonts
}{}
\makeatletter
\@ifundefined{KOMAClassName}{% if non-KOMA class
  \IfFileExists{parskip.sty}{%
    \usepackage{parskip}
  }{% else
    \setlength{\parindent}{0pt}
    \setlength{\parskip}{6pt plus 2pt minus 1pt}}
}{% if KOMA class
  \KOMAoptions{parskip=half}}
\makeatother
\usepackage{color}
\usepackage{fancyvrb}
\newcommand{\VerbBar}{|}
\newcommand{\VERB}{\Verb[commandchars=\\\{\}]}
\DefineVerbatimEnvironment{Highlighting}{Verbatim}{commandchars=\\\{\}}
% Add ',fontsize=\small' for more characters per line
\newenvironment{Shaded}{}{}
\newcommand{\AlertTok}[1]{\textcolor[rgb]{1.00,0.00,0.00}{\textbf{#1}}}
\newcommand{\AnnotationTok}[1]{\textcolor[rgb]{0.38,0.63,0.69}{\textbf{\textit{#1}}}}
\newcommand{\AttributeTok}[1]{\textcolor[rgb]{0.49,0.56,0.16}{#1}}
\newcommand{\BaseNTok}[1]{\textcolor[rgb]{0.25,0.63,0.44}{#1}}
\newcommand{\BuiltInTok}[1]{\textcolor[rgb]{0.00,0.50,0.00}{#1}}
\newcommand{\CharTok}[1]{\textcolor[rgb]{0.25,0.44,0.63}{#1}}
\newcommand{\CommentTok}[1]{\textcolor[rgb]{0.38,0.63,0.69}{\textit{#1}}}
\newcommand{\CommentVarTok}[1]{\textcolor[rgb]{0.38,0.63,0.69}{\textbf{\textit{#1}}}}
\newcommand{\ConstantTok}[1]{\textcolor[rgb]{0.53,0.00,0.00}{#1}}
\newcommand{\ControlFlowTok}[1]{\textcolor[rgb]{0.00,0.44,0.13}{\textbf{#1}}}
\newcommand{\DataTypeTok}[1]{\textcolor[rgb]{0.56,0.13,0.00}{#1}}
\newcommand{\DecValTok}[1]{\textcolor[rgb]{0.25,0.63,0.44}{#1}}
\newcommand{\DocumentationTok}[1]{\textcolor[rgb]{0.73,0.13,0.13}{\textit{#1}}}
\newcommand{\ErrorTok}[1]{\textcolor[rgb]{1.00,0.00,0.00}{\textbf{#1}}}
\newcommand{\ExtensionTok}[1]{#1}
\newcommand{\FloatTok}[1]{\textcolor[rgb]{0.25,0.63,0.44}{#1}}
\newcommand{\FunctionTok}[1]{\textcolor[rgb]{0.02,0.16,0.49}{#1}}
\newcommand{\ImportTok}[1]{\textcolor[rgb]{0.00,0.50,0.00}{\textbf{#1}}}
\newcommand{\InformationTok}[1]{\textcolor[rgb]{0.38,0.63,0.69}{\textbf{\textit{#1}}}}
\newcommand{\KeywordTok}[1]{\textcolor[rgb]{0.00,0.44,0.13}{\textbf{#1}}}
\newcommand{\NormalTok}[1]{#1}
\newcommand{\OperatorTok}[1]{\textcolor[rgb]{0.40,0.40,0.40}{#1}}
\newcommand{\OtherTok}[1]{\textcolor[rgb]{0.00,0.44,0.13}{#1}}
\newcommand{\PreprocessorTok}[1]{\textcolor[rgb]{0.74,0.48,0.00}{#1}}
\newcommand{\RegionMarkerTok}[1]{#1}
\newcommand{\SpecialCharTok}[1]{\textcolor[rgb]{0.25,0.44,0.63}{#1}}
\newcommand{\SpecialStringTok}[1]{\textcolor[rgb]{0.73,0.40,0.53}{#1}}
\newcommand{\StringTok}[1]{\textcolor[rgb]{0.25,0.44,0.63}{#1}}
\newcommand{\VariableTok}[1]{\textcolor[rgb]{0.10,0.09,0.49}{#1}}
\newcommand{\VerbatimStringTok}[1]{\textcolor[rgb]{0.25,0.44,0.63}{#1}}
\newcommand{\WarningTok}[1]{\textcolor[rgb]{0.38,0.63,0.69}{\textbf{\textit{#1}}}}
\usepackage{longtable,booktabs,array}
\newcounter{none} % for unnumbered tables
\usepackage{calc} % for calculating minipage widths
% Correct order of tables after \paragraph or \subparagraph
\usepackage{etoolbox}
\makeatletter
\patchcmd\longtable{\par}{\if@noskipsec\mbox{}\fi\par}{}{}
\makeatother
% Allow footnotes in longtable head/foot
\IfFileExists{footnotehyper.sty}{\usepackage{footnotehyper}}{\usepackage{footnote}}
\makesavenoteenv{longtable}
\usepackage{graphicx}
\makeatletter
\newsavebox\pandoc@box
\newcommand*\pandocbounded[1]{% scales image to fit in text height/width
  \sbox\pandoc@box{#1}%
  \Gscale@div\@tempa{\textheight}{\dimexpr\ht\pandoc@box+\dp\pandoc@box\relax}%
  \Gscale@div\@tempb{\linewidth}{\wd\pandoc@box}%
  \ifdim\@tempb\p@<\@tempa\p@\let\@tempa\@tempb\fi% select the smaller of both
  \ifdim\@tempa\p@<\p@\scalebox{\@tempa}{\usebox\pandoc@box}%
  \else\usebox{\pandoc@box}%
  \fi%
}
% Set default figure placement to htbp
\def\fps@figure{htbp}
\makeatother
\ifLuaTeX
  \usepackage{luacolor}
  \usepackage[soul]{lua-ul}
\else
  \usepackage{soul}
\fi
\setlength{\emergencystretch}{3em} % prevent overfull lines
\providecommand{\tightlist}{%
  \setlength{\itemsep}{0pt}\setlength{\parskip}{0pt}}
\usepackage{bookmark}
\IfFileExists{xurl.sty}{\usepackage{xurl}}{} % add URL line breaks if available
\urlstyle{same}
\hypersetup{
  pdftitle={Sous-Chef AI},
  pdfauthor={Nome Cognome matricola studente},
  hidelinks,
  pdfcreator={LaTeX via pandoc}}

\title{Sous-Chef AI}
\author{Nome Cognome matricola studente}
\date{a.a. 2025/2026}

\begin{document}
\maketitle

\pandocbounded{\includegraphics[keepaspectratio,alt={project-logo}]{C:/Users/pietro/edu_digital/assets/image-20260111125853833.png}}

\section{Sous-Chef AI}\label{sous-chef-ai}

Sistema automatizzato per la ricerca, la costruzione, la revisione ed il
rilascio di manuali d\textquotesingle uso.

\begin{quote}
Sous-Chef AI: il tuo brigadiere digitale, tu sei lo Chef!
\end{quote}

\subsection{Introduzione}\label{introduzione}

Il progetto ha due soggetti principali:

\begin{itemize}
\item
  Il \textbf{sistema}, ovvero il generatore di manuali
\item
  Il \textbf{manuale}, ovvero l\textquotesingle output del sistema
\end{itemize}

Il \textbf{sistema} mira a semplificare il processo di realizzazione di
manuali fornendo all\textquotesingle utenza:

\begin{itemize}
\item
  \emph{Ricerca automatizzata} all\textquotesingle interno delle fonti
  indicate
\item
  \emph{Traduzione e adattamento} del contenuto rispetto al pubblico di
  riferimento
\item
  \emph{Processo di revisione c}he permette la modifica del contenuto
  fornito o la sua completa riscrittura, specificando un prompt
  aggiuntivo
\item
  \emph{Pubblicazione automatica} su
  \href{https://roccobalocco.github.io/edu_digital/}{GitHubPages} di
  tutti i file generati dal sistema
\item
  \emph{Creazione del \textbf{manuale}}, composto da più sezioni
  generate e revisionate singolarmente, in vari formati (\emph{tex},
  \emph{html}, \emph{pdf}, \emph{epub}, \emph{md})
\end{itemize}

Il \textbf{manuale}, tramite linguaggio non tecnico e coerente con il
contesto aziendale, mira a spiegare e comprendere:

\begin{itemize}
\item
  \emph{Come utilizzare l\textquotesingle IA}
  all\textquotesingle interno dell\textquotesingle azienda
\item
  \emph{Come adattare istruzioni operative} ai diversi casi
  d\textquotesingle uso
\item
  \emph{Esempi pratici} contestualizzati
\item
  \emph{Best practices e limiti} dell\textquotesingle IA
\end{itemize}

Le tecnologie adottate sono le seguenti:

\begin{itemize}
\item
  \emph{Python 3.12.1} come linguaggio di programmazione
\item
  \emph{LangGraph} per modellare il flusso di generazione del contenuto
\item
  \emph{LangChain} come framework di orchestrazione
\item
  \emph{Google Gemini} per i modelli generativo
\item
  \emph{Vector store} basato su \emph{FAISS} e embedding semantici
\item
  \emph{GitHub Actions} per automatizzare:

  \begin{itemize}
  \item
    La fase di \emph{pubblicazione} dei file generati su \emph{GitHub
    Pages}
  \item
    La fase di \emph{generazione} in diversi formati utilizzando
    \emph{Pandoc}
  \end{itemize}
\end{itemize}

\subsection{\texorpdfstring{Ideazione }{Ideazione }}\label{ideazione}

\subsubsection{Tema}\label{tema}

Il \textbf{sistema} è stato progettato per essere quanto più generale
possibile, infatti le tematiche che esso ricerca sono elencate
all\textquotesingle interno di \texttt{prompts.py}, alla voce
\texttt{TOPIC\_SPECS}.

L\textquotesingle utenza può modellare le tematiche, e la loro
composizione, utilizzando il seguente schema:

\begin{Shaded}
\begin{Highlighting}[]
\StringTok{"topic\_title"}\ErrorTok{:} \FunctionTok{\{}

    \DataTypeTok{"title"}\FunctionTok{:} \StringTok{"Title to include"}\FunctionTok{,}

    \DataTypeTok{"query"}\FunctionTok{:} \StringTok{"Query for LLM"}\FunctionTok{,}

    \DataTypeTok{"deliverable\_sections"}\FunctionTok{:} \OtherTok{[}

        \StringTok{"Section1"}\OtherTok{,}

        \StringTok{"Section2"}\OtherTok{,}

        \StringTok{"Example1"}\OtherTok{,}

        \StringTok{"Example2"}\OtherTok{,}

        \StringTok{"Pros/Cons"}\OtherTok{,}

    \OtherTok{]}

\FunctionTok{\}}
\end{Highlighting}
\end{Shaded}

Il processo di selezione dei documenti rilevanti è
anch\textquotesingle esso personalizzabile tramite la voce
\texttt{TOPIC\_SELECTOR\_SYSTEM}, all\textquotesingle interno di
\texttt{prompts.py}, il quale rappresenta il prompt di sistema
utilizzato per filtrare gli estratti ottenuti dalle fonti mediante
FAISS.

I temi principali su cui il l\textquotesingle{}\textbf{utente
utilizzatore} si concentra diventano quindi:

\begin{itemize}
\item
  \emph{Una efficace scrittura dei prompt} , sia in fase di estrazione
  che di selezione
\item
  \emph{Il controllo e la revisione dei contenuti generati}, che
  consente un ulteriore fine tuning mediante prompt aggiuntivi
\end{itemize}

Mentre il \textbf{sistema} concentra l\textquotesingle attenzione su tre
aspetti principali:

\begin{itemize}
\item
  \emph{Automazione della produzione dei contenuti}, che velocizza la
  creazione del manuale e riduce il carico cognitivo dell'utente nella
  fase di ricerca.
\item
  \emph{Revisione umana,} necessaria per assicurare accuratezza,
  coerenza e aderenza alle specifiche editoriali.
\item
  \emph{Adattamento dei contenuti}, che consente di personalizzare il
  manuale in base ai feedback e alle esigenze specifiche di ciascun
  argomento.
\end{itemize}

Anche se la generazione dei contenuti risulta essere automatica, essa
\emph{richiede un intervento attivo da parte
dell\textquotesingle utenza} per adattare i temi alle specifiche
necessità del manuale. Ogni sezione, e quindi ogni "\emph{topic}", deve
essere sottoposta ad una revisione rigida e rigorosa, garantendo così la
coerenza e la qualità del sistema di conoscenze generato.

L\textquotesingle{}\textbf{utente} non deve più concentrarsi sulla
ricerca del contenuto perfetto, ma solo sulla valutazione e
sull\textquotesingle adattamento dei materiali proposti, dedicando più
tempo alla revisione e agli aspetti più importanti del processo, ovvero
seguire le specifiche richieste editoriali.

Il \textbf{manuale} invece si concentra su contenuti concreti e
fruibili, pensati per guidare i destinatari nell'uso dell'IA in contesti
reali.

I temi principali che il manuale affronta sono:

\begin{itemize}
\item
  \emph{Introduzione all'uso dell'IA in azienda}, fornendo una
  panoramica chiara del ruolo dell'intelligenza artificiale nel lavoro
  quotidiano di copywriter e content strategist, spiegandone i vantaggi
  ed i limiti
\item
  \emph{Istruzioni operative per casi d'uso specifici}, ogni sezione
  guida l'utente nella costruzione e nell'uso di prompt per attività
  come:

  \begin{itemize}
  \item
    \ul{Text generation}, creazione di contenuti testuali per post
    social, newsletter o articoli
  \item
    \ul{Language detection}, identificazione automatica della lingua dei
    contenuti
  \item
    \ul{Cross-tabular analysis}, analisi di tabelle e dati testuali per
    insight strategici
  \end{itemize}
\item
  \emph{Esempi pratici contestualizzati}, gli esempi sono selezionati
  dal repository OpenAI Cookbook, tradotti e adattati al contesto
  aziendale.
\item
  \emph{Best practices e limiti d'uso}, include linee guida per evitare
  errori comuni nell'uso dell'IA generativa, gestione dei bias, verifica
  dei contenuti generati e suggerimenti per l'integrazione dei
  risultati.
\end{itemize}

In sintesi, mentre il \textbf{sistema} si concentra sull'automazione, la
selezione e la generazione dei contenuti, il \textbf{manuale} offre una
guida pratica e contestualizzata che consente all'utente di applicare
efficacemente l'IA, mantenendo il controllo creativo e strategico sui
contenuti prodotti.

\subsubsection{Destinatari}\label{destinatari}

\textbf{Sous-Chef AI} ha lo scopo di supportare il creatore del manuale,
semplificando e automatizzando gran parte del flusso editoriale. Grazie
agli strumenti utilizzati, il creatore può indicizzare, selezionare,
adattare e revisionare contenuti a partire dal repository OpenAI
Cookbook, riducendo il carico cognitivo e concentrandosi sul controllo
qualitativo dei materiali.

Il \textbf{manuale} é pensato per copywriter e content strategist
all\textquotesingle interno di una agenzia di comunicazione digitale.
Esso fornisce linee guida pratiche, esempi contestualizzati e buone
pratiche nell\textquotesingle uso dell\textquotesingle IA generativa,
utilizzando un linguaggio accessibile e quanto più lontano dal tecnico.

\paragraph{Personas}\label{personas}

\pandocbounded{\includegraphics[keepaspectratio,alt={image-20260111154757334}]{C:/Users/pietro/edu_digital/assets/image-20260111154757334.png}}

\textbf{Scenario d\textquotesingle uso}: deve preparare un calendario
editoriale e utilizza il manuale per capire come generare contenuti
coerenti e adatti al tono aziendale.

\pandocbounded{\includegraphics[keepaspectratio,alt={}]{C:/Users/pietro/edu_digital/assets/image-20260111153938613.png}}

\textbf{Scenario d\textquotesingle uso}: deve identificare delle best
practices per generare testi coerenti e personalizzati, consultando il
manuale per selezionare/creare prompt efficaci e comprendere i limiti
degli strumenti AI.

\pandocbounded{\includegraphics[keepaspectratio,alt={}]{C:/Users/pietro/edu_digital/assets/image-20260111154509526.png}}

\textbf{Scenario d\textquotesingle uso}: usa il manuale per definire
standard condivisi, monitorare le pratiche del team e introdurre nuovi
strumenti in modo controllato.

\subsubsection{Requisiti di
accettazione}\label{requisiti-di-accettazione}

Per raggiungere efficacemente i rispettivi destinatari, il
\textbf{manuale} e il \textbf{sistema}, devono soddisfare una serie di
richieste, le quali spaziano su diverse aree.

\paragraph{Requisiti funzionali}\label{requisiti-funzionali}

\begin{itemize}
\item
  \emph{Completezza dei contenuti}, il manuale deve includere istruzioni
  operative, esempi e linee guida sui limiti d\textquotesingle uso,
  coprendo almeno i casi d\textquotesingle uso selezionati
\item
  \emph{Chiarezza e comprensibilità}, il linguaggio deve essere fruibile
  da personale non tecnico e deve rimanere coerente con il contesto
  aziendale
\item
  \emph{Modularità}, ogni sezione deve risultare indipendente e
  integrabile senza compromettere la coerenza complessiva
\item
  \emph{Manutenibilità}, il sistema deve consentire la generazione di
  nuove versioni del manuale con facilità, sfruttando il flusso
  automatizzato
\item
  \emph{Validazione umana}, tutte le sezioni devono passare attraverso
  un processo di revisione editoriale, atto a garantire accuratezza,
  coerenza e aderenza alle specifiche
\item
  \emph{mdBook}, per pubblicare richiede

  \begin{itemize}
  \item
    un file \texttt{book.toml} dove indicare il titolo,
    l\textquotesingle autore, la fonte del libro e la cartella di output
  \item
    un file \texttt{SUMMARY.md} dove indicare l\textquotesingle indice
    del libro
  \end{itemize}
\end{itemize}

\paragraph{Modelli di fruizione}\label{modelli-di-fruizione}

\begin{itemize}
\item
  \emph{Lettura modulare}, i destinatari possono accedere direttamente
  alle sezioni pertinenti ai loro compiti, senza dover leggere
  l\textquotesingle intero manuale
\item
  \emph{Accesso multi canale/formato}, il manuale deve essere
  disponibili nel maggior numero di formati
\end{itemize}

\paragraph{Innovazione}\label{innovazione}

\begin{itemize}
\item
  \emph{Human in the loop}, combinazione di generazione automatica con
  revisione umana per massimizzare efficienza e qualità
\item
  \emph{Personalizzazione}, possibilità, per l\textquotesingle utenza
  del sistema, di adattare prompt e sezioni ai casi
  d\textquotesingle uso specifici, creando manuali su misura
\item
  \emph{Automazione del flusso editoriale}, gestione automatica di
  indicizzazione, selezione e pubblicazione dei contenuti, riducendo in
  questo modo il carico operativo
\item
  \emph{Flessibilità nei canali di distribuzione}, generazione di output
  multi formato compatibili con web, intranet, stampa e multi
  dispositivo, semplificandone la diffusione
\end{itemize}

\subsubsection{Canali di distribuzione}\label{canali-di-distribuzione}

Il \textbf{manuale} d'uso generato da \textbf{Sous-Chef AI} è pensato
per essere fruibile su più canali, in base alle esigenze dei
destinatari.

\paragraph{Canali principali:}\label{canali-principali}

{\def\LTcaptype{none} % do not increment counter
\begin{longtable}[]{@{}llll@{}}
\toprule\noalign{}
Canale & Descrizione & Formati & Note \\
\midrule\noalign{}
\endhead
\bottomrule\noalign{}
\endlastfoot
\textbf{Web/Intranet} & Manuale accessibile via intranet, portale
aziendale o link pubblico & HTML, PDF, md (per mdBook) & Navigazione
modulare con link interni e indici interattivi, ricerca integrata
(mdBook) \\
\textbf{Social} & Sintesi o estratti mirati per formazione rapida & PDF,
snippet & Contenuti ridotti per fruizione veloce, non completo \\
\textbf{Repository/Marketplace} & Condivisione con altri team o agenzie
& Markdown, TeX & Versione ``template'' riutilizzabile per progetti
futuri \\
\end{longtable}
}

\paragraph{Formati:}\label{formati}

Il \textbf{manuale} risulta disponibile nei seguenti formati per
garantire massima flessibilità:

\begin{itemize}
\item
  \textbf{\href{https://github.com/roccobalocco/edu_digital/releases/download/docs-20260110175016/manuale_completo.md}{Markdown}}:
  formato sorgente, sia via moduli sia monolitico, facilmente
  aggiornabile e riutilizzabile.
\item
  \textbf{\href{https://github.com/roccobalocco/edu_digital/releases/download/docs-20260110175016/manuale_completo.html}{HTML}
  / \href{https://roccobalocco.github.io/edu_digital/}{mdBook}}:
  fruizione online con navigazione interattiva.
\item
  \textbf{\href{https://github.com/roccobalocco/edu_digital/releases/download/docs-20260110175016/manuale_completo.pdf}{PDF}}:
  stampa e distribuzione offline, con layout coerente e professionale.
\item
  \textbf{\href{https://github.com/roccobalocco/edu_digital/releases/download/docs-20260110175016/manuale_completo.epub}{ePub}}:
  lettura su dispositivi mobili e reader digitali.
\item
  \textbf{\href{https://github.com/roccobalocco/edu_digital/releases/download/docs-20260110175016/manuale_completo.tex}{Tex
  / LaTeX}}: generazione di versioni professionali tipograficamente
  accurate.
\end{itemize}

\paragraph{Identità visuale}\label{identituxe0-visuale}

Il \textbf{manuale} si concentra sulla leggibilità e sulla facilità di
gestione nel tempo; per questo motivo non contiene alcuna
personalizzazione visuale.

La responsabilità di definire l'identità visiva è lasciata agli
utilizzatori del \textbf{sistema}. In futuro sarà possibile esportare i
contenuti applicando stili, template, layout e altre personalizzazioni
grazie a Pandoc.

Il formato Markdown garantisce una base solida e coerente per la
formattazione, che può poi essere adattata liberamente dall'utente.

\paragraph{Tradizione vs Innovazione}\label{tradizione-vs-innovazione}

Il \textbf{manuale} mantiene coerenza con modelli consolidati di
documentazione digitale (ad esempio guide operative, manuali tecnici in
md/pdf/html). Allo stesso tempo il \textbf{sistema} introduce elementi
innovativi, come workflow automatizzati, human-in-the-loop e
personalizzazione dei contenuti, per offrire una fruizione più dinamica
e modulare.

\subsection{Processo di Produzione}\label{processo-di-produzione}

Il processo di produzione del \textbf{manuale} combina l'acquisizione
dei contenuti con la loro gestione strutturata, garantendo coerenza,
tracciabilità e aggiornabilità.

Il \textbf{sistema} è progettato per utilizzare qualsiasi fonte in
formato Jupyter Notebook (\texttt{.ipynb}). Per cambiare la fonte, basta
modificare il percorso della variabile di ambiente
\texttt{COOKBOOK\_PATH}. Per il primo manuale, il percorso era impostato
sul repository OpenAI Cookbook, ma può essere adattato a qualsiasi altra
raccolta di notebook.

Per quanto riguarda il \textbf{manuale}, le fonti attraversano tre fasi
principali di trasformazione:

\begin{enumerate}
\def\labelenumi{\arabic{enumi}.}
\item
  Inizialmente sono \emph{file di OpenAI Cookbook}, disponibili in
  formato \texttt{.ipynb}
\item
  Successivamente divengono \emph{contenuti generati automaticamente},
  selezionati e adattati tramite LLM in sezioni coerenti con il contesto
  aziendale
\item
  Infine, grazie ad una fase di lavoro manuale, necessario in alcuni
  casi per traduzione, revisione o adattamento dei contenuti, arrivano
  ad essere \emph{sezioni del manuale}
\end{enumerate}

Il flusso di gestione documentale segue queste fasi principali:

\begin{enumerate}
\def\labelenumi{\arabic{enumi}.}
\item
  \emph{Indicizzazione e raccolta dei contenuti}, le fonti vengono
  organizzate in un vector store (FAISS) per un recupero rapido e mirato
\item
  \emph{Selezione dei documenti rilevanti}, tramite prompt guidati e
  criteri definiti dall'utente
\item
  \emph{Adattamento e generazione delle sezioni}, trasformazione dei
  contenuti in sezioni Markdown coerenti e leggibili
\item
  \emph{Revisione editoriale}, modifica, approvazione o rigenerazione
  delle sezioni basata su feedback dell'utente
\item
  \emph{Commit e versionamento}, le sezioni approvate vengono salvate
  tramite git una volta fatto il commit
\item
  \emph{Pubblicazione e esportazione}, tramite GitHub Actions, il
  contenuto viene distribuito su GitHub Pages e reso disponibile in
  diversi formati (HTML, PDF, ePub, TeX, Markdown)
\end{enumerate}

Il \textbf{sistema} ed il \textbf{manuale} possono vivere nello stesso
ambiente, pertanto il versionamento ricade sullo stesso \emph{git},
oltre che al tag applicato dalla GitHub Action che provvede al rilascio.

Il seguente diagramma mostra graficamente come si articola il flusso di
produzione, di gestione e di pubblicazione dei contenuti:

\includegraphics[width=4.59375in,height=\textheight,keepaspectratio,alt={}]{17681502761611.png}

\subsubsection{Tecnologie adottate}\label{tecnologie-adottate}

Il progetto integra diverse tecnologie, ciascuna con un ruolo specifico
nelle fasi di produzione del manuale, al fine di garantire efficienza,
coerenza e qualità dei contenuti.

\begin{itemize}
\item
  \emph{Python 3.12.1}, linguaggio di programmazione principale per lo
  sviluppo del sistema
\item
  \emph{LangGraph}, utilizzato per modellare il flusso di generazione
  dei contenuti (indicizzazione, recupero, selezione, adattamento e
  revisione). Contribuisce a rendere il processo \emph{modulare,
  tracciabile e ripetibile}, riducendo errori e ridondanze nella
  produzione
\item
  \emph{LangChain}, framework di orchestrazione dei LLM. Coordina
  l'interazione tra il sistema di gestione dei documenti e il modello
  generativo, supportando l'automazione delle fasi di selezione e
  adattamento dei contenuti
\item
  \emph{Google} \emph{Gemini}, modello generativo impiegato per la
  trasformazione dei contenuti selezionati in sezioni leggibili e
  coerenti con il contesto aziendale

  \begin{itemize}
  \item
    \texttt{env.GEMINI\_MODEL}, utilizzato per la generazione dei testi
    (\ul{gemini-2.5-flash})
  \item
    \texttt{env.GEMINI\_EMBED\_MODEL}, utilizzato per creare embedding
    semantici delle fonti (t\ul{ext-embedding-004})
  \end{itemize}

  Contribuisce a \emph{velocizzare la produzione}, mantenendo un
  linguaggio non tecnico e adatto all'utenza
\item
  \emph{Vector store con FAISS ed embedding semantici}, permette
  l'indicizzazione delle fonti e il recupero rapido dei documenti più
  rilevanti. Prova a garantire che i contenuti generati siano
  \emph{accurati e contestualizzati}
\item
  \emph{GitHub Actions}, automatizzano le fasi di commit, pubblicazione
  e generazione del manuale in diversi formati (HTML, PDF, ePub, TeX,
  Markdown) mediante i seguenti workflow:

  \begin{itemize}
  \item
    \href{https://github.com/roccobalocco/edu_digital/blob/docs-20260110175016/.github/workflows/conversion.yml}{conversion.yml},
    workflow che converte il manuale nei formati di destinazione,
    utilizzando \ul{Pandoc}, per poi crearne un rilascio su \ul{GitHub}
  \item
    \href{https://github.com/roccobalocco/edu_digital/blob/docs-20260110175016/.github/workflows/mdbook.yml}{mdbook.yml},
    workflow che aggiorna la GitHub Page legata al repository
  \end{itemize}

  Contribuiscono a rendere il flusso \emph{efficiente, ripetibile e
  facilmente aggiornabile}, supportando la distribuzione multi-canale
  dei contenuti
\end{itemize}

\subsubsection{Esecuzione del flusso}\label{esecuzione-del-flusso}

\href{https://github.com/roccobalocco/edu_digital}{Repository}.

Per eseguire il progetto si deve:

\begin{itemize}
\item
  \emph{copiare ed adattare il file} \texttt{.env.example} in
  \texttt{.env}, mettendo la propria chiave API ed il proprio percorso
  al suo interno
\item
  \emph{installare tutte le dipendenze} elencate in
  \texttt{requirements.txt}
\item
  \emph{avere un documentale} (in questo caso il OpenAI Cookbook)
  all\textquotesingle interno del percorso specificato
\item
  \emph{eseguire il comando} \texttt{python\ run.py} con
  l\textquotesingle opzione \texttt{-\/-full} per avere un file
  comprensivo di tutte le sezioni
\end{itemize}

\subsubsection{Utilizzo di IA
generativa}\label{utilizzo-di-ia-generativa}

L' IA generativa è stata integrata in più fasi del flusso di gestione
documentale, con l'obiettivo di automatizzare, accelerare e rendere
scalabile la produzione del manuale, mantenendo elevati standard di
qualità.

\paragraph{Fasi in cui l'IA è stata
applicata}\label{fasi-in-cui-lia-uxe8-stata-applicata}

\begin{enumerate}
\def\labelenumi{\arabic{enumi}.}
\item
  \emph{Selezione dei documenti rilevanti}

  \begin{itemize}
  \item
    Viene utilizzato un modello di linguaggio per analizzare gli
    estratti provenienti dal vector store (FAISS) e identificare i
    contenuti più pertinenti per ciascun topic
  \item
    Obiettivo: \ul{ridurre il tempo che l'utente impiegherebbe a leggere
    e filtrare} manualmente centinaia di notebook o estratti
  \end{itemize}
\item
  \emph{Adattamento e generazione delle sezioni}

  \begin{itemize}
  \item
    L'IA genera sezioni del manuale in linguaggio coerente e non
    tecnico, adattando i contenuti selezionati al contesto aziendale
  \item
    Obiettivo: \ul{produrre rapidamente contenuti} leggibili, coerenti e
    pronti per la revisione.
  \end{itemize}
\item
  \emph{Rigenerazione basata su feedback}

  \begin{itemize}
  \item
    Durante la fase di revisione, l'utente può fornire indicazioni
    aggiuntive (prompt) che vengono utilizzate dal modello per
    migliorare o riformulare il contenuto
  \item
    Obiettivo: \ul{integrare il cosiddetto human-in-the-loop,}
    combinando automazione e intervento umano per assicurare accuratezza
    e aderenza alle specifiche
  \end{itemize}
\end{enumerate}

\paragraph{Approccio di prompt
engineering}\label{approccio-di-prompt-engineering}

I prompt sono suddivisi in due macro categorie:

\begin{enumerate}
\def\labelenumi{\arabic{enumi}.}
\item
  \emph{Prompt di selezione} (\texttt{TOPIC\_SELECTOR\_SYSTEM}): guidano
  l'IA nell'identificazione degli estratti più pertinenti.
\item
  \emph{Prompt di generazione/adattamento} (\texttt{ADAPT\_SYSTEM}):
  definiscono struttura, stile e contenuti delle sezioni del manuale.
\end{enumerate}

Tutti i prompt sono personalizzabili dall'utente utilizzatore del
sistema, consentendo un fine-tuning continuo in base ai feedback
editoriali.

\paragraph{Validazione e controllo
qualità}\label{validazione-e-controllo-qualituxe0}

Tutti gli output generati dall'IA passano attraverso una fase di
revisione umana. L'utente può approvare, modificare o rigenerare ogni
sezione, garantendo:

\begin{itemize}
\item
  \emph{Coerenza} tra i materiali
\item
  \emph{Accuratezza} delle informazioni
\item
  \emph{Aderenza} al contesto aziendale e alle specifiche richieste
\end{itemize}

\paragraph{Contributo dell'IA}\label{contributo-dellia}

\subparagraph{Riduzione dei tempi}\label{riduzione-dei-tempi}

La selezione automatica dei contenuti e la generazione delle sezioni
riducono drasticamente il lavoro manuale.

\subparagraph{Miglioramento della
qualità}\label{miglioramento-della-qualituxe0}

L'IA fornisce una base coerente e leggibile, su cui la revisione umana
può concentrarsi sul fine-tuning e sugli aspetti strategici.

\subparagraph{Scalabilità}\label{scalabilituxe0}

Il sistema può elaborare grandi volumi di contenuti in tempi contenuti,
adattandosi facilmente a nuovi dataset o fonti.

\paragraph{Limiti e intervento umano}\label{limiti-e-intervento-umano}

L'IA non è in grado di garantire completezza o correttezza totale delle
informazioni. La revisione umana rimane essenziale per:

\begin{itemize}
\item
  \emph{Contestualizzare} i contenuti al target aziendale
\item
  \emph{Adattare} esempi pratici
\item
  \emph{Validare} coerenza e accuratezza
\end{itemize}

\begin{itemize}
\item
  L'approccio \emph{human-in-the-loop} assicura un equilibrio tra
  automazione e controllo editoriale.
\end{itemize}

\subsection{Valutazione dei risultati
raggiunti}\label{valutazione-dei-risultati-raggiunti}

\subsubsection{Valutazione del flusso di
produzione}\label{valutazione-del-flusso-di-produzione}

Il flusso di produzione implementato con \textbf{Sous-Chef AI} ha
permesso di ottenere benefici misurabili nelle diverse fasi del ciclo
documentale:

\begin{enumerate}
\def\labelenumi{\arabic{enumi}.}
\item
  \emph{Riduzione dei tempi di gestione documentale} grazie alla
  selezione automatica dei documenti tramite FAISS e la generazione di
  sezioni con Google Gemini
\item
  \emph{Riduzione degli errori}, la struttura modulare, il versionamento
  e la revisione guidata dall'utente limitano gli errori di coerenza e
  di contenuto, assicurando che ogni sezione sia correttamente
  contestualizzata
\item
  \emph{Miglioramento della qualità dei documenti}, l'uso combinato di
  LLM e revisione umana dovrebbe garantire testi leggibili e coerenti
  con il contesto aziendale
\item
  \emph{Miglioramento del livello di accettazione della tecnologia}, la
  possibilità di personalizzare i prompt e gestire i contenuti secondo
  le proprie esigenze dovrebbe rendere il sistema più intuitivo e
  facilmente adottabile
\item
  \emph{Raggiungimento di nuovi canali di distribuzione}, l'esportazione
  automatica dei contenuti, unita alla pubblicazione su GitHub Pages, ha
  permesso di raggiungere facilmente più canali
\item
  \emph{Soddisfacimento di nuovi scenari d'uso}, il sistema supporta
  scenari non previsti inizialmente, come l'adattamento rapido di
  contenuti a nuovi dataset o l'integrazione di fonti alternative,
  grazie alla flessibilità della pipeline
\end{enumerate}

\subsubsection{Confronto con lo stato
dell\textquotesingle arte}\label{confronto-con-lo-stato-dellarte}

\paragraph{ASIS (flusso tradizionale)}\label{asis-flusso-tradizionale}

Raccolta manuale dei contenuti, selezione e revisione esclusivamente
umane, trasformazione dei formati effettuata singolarmente.

\subparagraph{\texorpdfstring{Problemi principali
}{Problemi principali }}\label{problemi-principali}

\begin{itemize}
\item
  \emph{Tempi lunghi}
\item
  \emph{Alto rischio di incoerenza tra sezioni}
\item
  \emph{Limitata scalabilità}
\item
  \emph{Difficoltà nel mantenere aggiornamenti frequenti}
\end{itemize}

\paragraph{TOBE (Sous-Chef AI)}\label{tobe-sous-chef-ai}

Pipeline automatizzata per selezione, adattamento e pubblicazione, con
revisione umana integrata.

\subparagraph{Vantaggi rilevanti}\label{vantaggi-rilevanti}

\begin{itemize}
\item
  \emph{Riduzione dei tempi nelle fasi di ricerca e generazione}
\item
  \emph{Maggiore coerenza}
\item
  \emph{Modularità del contenuto}
\item
  \emph{Possibilità di pubblicare su più formati e canali}
\end{itemize}

\subsubsection{Limiti emersi}\label{limiti-emersi}

Sono stati identificati i seguenti Limiti:

\begin{itemize}
\item
  \emph{Accesso limitato ad alcune tecnologie}, alcuni LLM o embedding
  più avanzati potrebbero non essere disponibili o richiedere costi
  aggiuntivi, che potrebbero risultare insostenibili
\item
  \emph{Automazione parziale dei formati}, la trasformazione completa in
  alcuni formati (es. TeX complesso o PDF con layout avanzato) può
  richiedere intervento manuale
\item
  Poco controllo sul layout, al momento non esiste un modo per gestire
  il layout del prodotto finale, se non intervenendo manualmente sul
  processo di conversione/generazione
\item
  \emph{Integrazione di fonti eterogenee}, il sistema è utilizzabile
  solo per fonti notebook \texttt{.ipynb}; formati diversi necessitano
  di adattamento preliminare
\end{itemize}

Gli ultimi due punti sono facilmente risolvibili
all\textquotesingle interno di eventuali nuove versioni di
\textbf{Sous-Chef AI}.

\subsection{Conclusioni}\label{conclusioni}

L'implementazione del \textbf{sistema} ha permesso di raggiungere gli
obiettivi principali definiti dai casi d'uso:

\begin{itemize}
\item
  \emph{Produzione rapida e modulare} di contenuti coerenti
\item
  \emph{Riduzione dei tempi} e degli errori nella gestione documentale
\item
  \emph{Maggiore scalabilità} e possibilità di raggiungere più canali di
  distribuzione
\end{itemize}

I risultati più soddisfacenti riguardano efficienza e modularità della
produzione, grazie all\textquotesingle IA generativa e alla gestione
automatizzata dei contenuti.

Le principali limitazioni restano legate alla alla gestione di fonti non
standard e alle trasformazioni avanzate di formato.

Tuttavia, il prototipo (interpretabile come un Proof of Concept)
realizzato dimostra chiaramente la fattibilità dell'approccio ed offre
un'ottima base per futuri miglioramenti e ampliamenti della pipeline
editoriale.

\subsection{Bibliografia, sitografia e strumenti
utilizzati}\label{bibliografia-sitografia-e-strumenti-utilizzati}

\begin{itemize}
\item
  \href{https://chatgpt.com/}{ChatGPT} e
  \href{https://claude.ai/}{Claude} come supporto nella stesura del
  codice e nella validazione di questo report
\item
  Know-how ottenuto implementando
  \href{https://www.softeam.it/prodotti/sofia/}{SofIA} (LangGraph,
  embedding, Python, CD/CI, etc...)
\item
  \href{https://github.com/features/codespaces}{GitHub Codespaces} per
  lo sviluppo in cloud
\item
  \href{https://app.creately.com/}{Creately} per la creazione delle
  personas cards
\item
  \href{github.com/roccobalocco/edu_digital/}{GitHub} per il repository
\item
  \href{https://sora.chatgpt.com}{Sora} per il logo del progetto
\end{itemize}

\end{document}
